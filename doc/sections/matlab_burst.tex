% =============================================================================
\section{MATLAB Burst I/O FFT Analysis}
% =============================================================================
\label{sec:burst_analysis}

This section analyzes the performance of the Burst I/O FFT using input
stimulus with known FFT bit-growth properties. Input stimulus used in
the following sections includes: impulse inputs, constant (DC) inputs,
complex-valued exponential inputs, uniform noise,
Gaussian noise, bandpass noise, and notch-filtered noise.
The FFT was configured for 18-bit data and 18-bit twiddles.

% -----------------------------------------------------------------------------
\subsection{Radix-4 Burst I/O FFT Model}
% -----------------------------------------------------------------------------

The Altera Burst I/O FFT is based on a radix-4 decimation-in-frequency (DIF)
architecture with integer input and block floating-point (BFP) output, i.e.,
the input uses real and imaginary integer values, while the output uses
real and imaginary integer values along with an exponent indicating the
number of right-shifts of the data that occurred during FFT
processing~\cite{Altera_FFT_UG_2017}.

The Altera FFT MATLAB model was used to create the Altera FFT spectra in this
document. The Altera FFT spectra have signal-to-quantization noise floor that
is 12dB (2-bits) lower than expected. Since the Altera FFT MATLAB source code
is not provided, the cause of this loss in signal-to-noise noise is not clear.
A radix-4 block floating-point FFT model was developed to compare to the
Altera FFT model.

The maximum scaling of a radix-4 butterfly is 4, and the maximum scaling of
the twiddle stage is $\sqrt{2}$, so the maximum total scaling is $4\sqrt{2}$
or a bit-growth of $\log_2(4\sqrt{2}) = 2.5$-bits. A radix-4 block
floating-point FFT operates by monitoring the 3 most-significant magnitude
bits of the real and imaginary components at each FFT stage.
%
When the FFT data is loaded into memory, one complex-valued sample per clock
period over $N$ clock periods, the 3 most-significant magnitude bits on
each real and imaginary value are compared to the respective sign-bit, and the
location of the highest magnitude bit is tracked as a scale factor to apply on
the \emph{next pass}.
%
When the FFT is processing data, 4 complex-valued samples are processed
per clock period over $N/4$ clock periods. The FFT data first passes through
a scaling stage, where the scale factor determined on the \emph{previous pass}
is applied The data then passes through the radix-4 and twiddle multipliers,
and again, the 3 most-significant magnitude bits are compared to the sign-bit,
and the location of the highest magnitude bit is tracked as a scale factor to
apply on the \emph{next pass}.

As an example, consider the FFT of 18-bit data with 18-bit twiddles. The input
data can use all 18-bits, eg., the 18-bit real or imaginary signed input values
can have the form
%
\begin{center}
\texttt{sX\_XXXX\_XXXX\_XXXX\_XXXX}
\end{center}
%
and as that data is written to memory, the FFT control logic determines the
scale factor required on the next pass. The data read from memory passes
through the scaling logic, and is scaled (right-shifted) to clear the
3 most-significant magitude bits, creating 18-bit values with 4
most-significant-bits (MSBs) of repeated sign bits, i.e.,
%
\begin{center}
\texttt{ss\_ssXX\_XXXX\_XXXX\_XXXX}
\end{center}
%
As this data passes through the radix-4 and twiddle stages, the FFT control logic
determines the scale factor to apply on the next pass.
%
The exponent is incremented by the scale factor applied during each FFT stage,
i.e., by 0, 1, 2, or 3. The exponent code tracks the total
number of shifts applied by the scaling stage. The scale factor calculated
during the last FFT stage is not applied and is not added to the exponent code.
The FFT output data may have 1, 2, 3, or 4 repeated sign-bits.

The scaling (right-shifting) logic and twiddle multiplier \emph{must} be
rounded  correctly, eg., truncation should not be used, as it introduces
a DC bias (offset) into the data.
%
Figure~\ref{fig:radix4_bfp_bandpass_truncate} shows how truncation DC bias
results in a DC spike (and noise plateaus) in the spectra (FFT output).

The quantization noise floor for a radix-4 BFP FFT is 3.5-bits above the
quantization noise floor for the input bit-width, i.e., for an 18-bit input,
the quantization noise floor when BFP exponent codes are being exercised
is 3-bits less due to the BFP encoding, and another 0.5-bit less due to
quantization (rounding) in the scaling and twiddle logic.

\clearpage
% -----------------------------------------------------------------------------
% Bandpass noise spectra for 18-bit
% -----------------------------------------------------------------------------
%
\begin{figure}[p]
  \begin{center}
    \includegraphics[width=0.8\textwidth]
    {figures/radix4_bfp_bandpass_18_18_n18dB_scale_truncate.pdf}\\
    (a) Scale truncation
  \end{center}
  \hfil
  \begin{center}
    \includegraphics[width=0.8\textwidth]
    {figures/radix4_bfp_bandpass_18_18_n18dB_product_truncate.pdf}\\
    (b) Product truncation
  \end{center}
  \caption{Bandpass noise spectra with DC spikes due to truncation.}
  \label{fig:radix4_bfp_bandpass_truncate}
\end{figure}
%
% -----------------------------------------------------------------------------

\clearpage
% -----------------------------------------------------------------------------
% Impulse phasors
% -----------------------------------------------------------------------------
%
\begin{figure}[t]
  \begin{center}
    \includegraphics[width=0.5\textwidth]
    {figures/altera_burst_impulse_18_18_impulses.pdf}
  \end{center}
  \caption{Impulse phasors.}
  \label{fig:impulse_phasors}
\end{figure}
%
% -----------------------------------------------------------------------------
%
% -----------------------------------------------------------------------------
\subsection{Impulse Input}
% -----------------------------------------------------------------------------

The FFT of an impulse generates a constant magnitude FFT response. If the
impulse is injected into input sample $x[0]$, then the FFT response for
each channel $X[k]$ is the same as $x[0]$, eg.,
%
\begin{verbatim}
>> N=8;x=zeros(N,1);x(1)=1-1j;X=fft(x);unique(X)
1.0000 - 1.0000i
\end{verbatim}
%
If the impulse is injected into input sample $x[1]$, then the FFT response
is a period of a complex-valued exponential, eg.,
%
\begin{verbatim}
>> N=8;x=zeros(N,1);x(2)=1-1j;X=fft(x);X
X =
   1.0000 - 1.0000i
   0.0000 - 1.4142i
  -1.0000 - 1.0000i
  -1.4142 + 0.0000i
  -1.0000 + 1.0000i
   0.0000 + 1.4142i
   1.0000 + 1.0000i
   1.4142 + 0.0000i
\end{verbatim}
%
Injecting an impulse into input sample $x[1]$ exercises more
features of the radix-4 butterfly and twiddle logic.
%
Figure~\ref{fig:impulse_phasors} shows the sixteen impulse phasors used
as Altera FFT input stimulus.

Figure~\ref{fig:altera_burst_impulse} shows the Altera Burst I/O FFT
frequency responses for the sixteen impulses, while
Figure~\ref{fig:radix4_bfp_impulse} shows the Radix-4 BFP FFT
frequency responses.
%
Figures~\ref{fig:altera_burst_impulse_error}
and~\ref{fig:radix4_bfp_impulse_error} show polar plots of the error
(difference between the MATLAB FFT) for the first six impulses.
%
The polar plots imply that the Radix-4 BFP error is larger, however,
that is an artifact of the normalization. The error plots were generated
as the error relative to the real or imaginary integer values by
scaling the MATLAB FFT by the exponent code.
%
The Altera Burst I/O FFT exponent codes were 5 and 6, whereas the
Radix-4 BFP FFT exponent codes were 3 and 4, i.e., 2-bits or a scale
factor of 4 smaller, so the Radix-4 BFP error is 2.43/4.00 = 0.61 on
the same scale as the Altera Burst I/O FFT.

\clearpage
% -----------------------------------------------------------------------------
% Altera Burst FFT Impulse Responses
% -----------------------------------------------------------------------------
%
\begin{figure}[p]
  \begin{minipage}{0.55\textwidth}
    \begin{center}
    \includegraphics[width=\textwidth]
    {figures/altera_burst_impulse_18_18_real.pdf}\\
    \vskip1mm
    \includegraphics[width=\textwidth]
    {figures/altera_burst_impulse_18_18_imag.pdf}\\
    (a) Impulse response versus channel
    \end{center}
  \end{minipage}
  \hfil
  \begin{minipage}{0.45\textwidth}
    \begin{center}
    \includegraphics[width=\textwidth]
    {figures/altera_burst_impulse_18_18_polar.pdf}\\
    (b) Impulse response polar
    \end{center}
  \end{minipage}
  \vskip5mm
  \begin{minipage}{0.55\textwidth}
    \begin{center}
    \includegraphics[width=\textwidth]
    {figures/altera_burst_impulse_18_18_error_real.pdf}\\
    \vskip2mm
    \includegraphics[width=\textwidth]
    {figures/altera_burst_impulse_18_18_error_imag.pdf}\\
    (c) Impulse response error versus channel
    \end{center}
  \end{minipage}
  \hfil
  \begin{minipage}{0.45\textwidth}
    \begin{center}
    \includegraphics[width=\textwidth]
    {figures/altera_burst_impulse_18_18_error_polar.pdf}\\
    (d) Impulse response error polar
    \end{center}
  \end{minipage}
  \caption{Altera Burst I/O FFT impulse response.}
  \label{fig:altera_burst_impulse}
\end{figure}
%
% -----------------------------------------------------------------------------

% -----------------------------------------------------------------------------
% Radix-4 BFP FFT Impulse Responses
% -----------------------------------------------------------------------------
%
\begin{figure}[p]
  \begin{minipage}{0.55\textwidth}
    \begin{center}
    \includegraphics[width=\textwidth]
    {figures/radix4_bfp_impulse_18_18_real.pdf}\\
    \vskip1mm
    \includegraphics[width=\textwidth]
    {figures/radix4_bfp_impulse_18_18_imag.pdf}\\
    (a) Impulse response versus channel
    \end{center}
  \end{minipage}
  \hfil
  \begin{minipage}{0.45\textwidth}
    \begin{center}
    \includegraphics[width=\textwidth]
    {figures/radix4_bfp_impulse_18_18_polar.pdf}\\
    (b) Impulse response polar
    \end{center}
  \end{minipage}
  \vskip5mm
  \begin{minipage}{0.55\textwidth}
    \begin{center}
    \includegraphics[width=\textwidth]
    {figures/radix4_bfp_impulse_18_18_error_real.pdf}\\
    \vskip2mm
    \includegraphics[width=\textwidth]
    {figures/radix4_bfp_impulse_18_18_error_imag.pdf}\\
    (c) Impulse response error versus channel
    \end{center}
  \end{minipage}
  \hfil
  \begin{minipage}{0.45\textwidth}
    \begin{center}
    \includegraphics[width=\textwidth]
    {figures/radix4_bfp_impulse_18_18_error_polar.pdf}\\
    (d) Impulse response error polar
    \end{center}
  \end{minipage}
  \caption{Radix-4 BFP FFT impulse response.}
  \label{fig:radix4_bfp_impulse}
\end{figure}
%
% -----------------------------------------------------------------------------

% -----------------------------------------------------------------------------
% Altera Burst FFT Polar Error Responses
% -----------------------------------------------------------------------------
%
\begin{figure}[p]
  \begin{minipage}{0.5\textwidth}
    \begin{center}
    \includegraphics[width=0.8\textwidth]
    {figures/altera_burst_impulse_18_18_error_polar_1.pdf}\\
    (a) Impulse (1.00, 0.00)
    \end{center}
  \end{minipage}
  \hfil
  \begin{minipage}{0.5\textwidth}
    \begin{center}
    \includegraphics[width=0.8\textwidth]
    {figures/altera_burst_impulse_18_18_error_polar_2.pdf}\\
    (b) Impulse (1.00, -0.41)
    \end{center}
  \end{minipage}
  \vskip5mm
  \begin{minipage}{0.5\textwidth}
    \begin{center}
    \includegraphics[width=0.8\textwidth]
    {figures/altera_burst_impulse_18_18_error_polar_3.pdf}\\
    (c) Impulse (1.00, -1.00)
    \end{center}
  \end{minipage}
  \hfil
  \begin{minipage}{0.5\textwidth}
    \begin{center}
    \includegraphics[width=0.8\textwidth]
    {figures/altera_burst_impulse_18_18_error_polar_4.pdf}\\
    (d) Impulse (0.41, -1.00)
    \end{center}
  \end{minipage}
  \vskip5mm
  \begin{minipage}{0.5\textwidth}
    \begin{center}
    \includegraphics[width=0.8\textwidth]
    {figures/altera_burst_impulse_18_18_error_polar_5.pdf}\\
    (e) Impulse (0.00, -1.00)
    \end{center}
  \end{minipage}
  \hfil
  \begin{minipage}{0.5\textwidth}
    \begin{center}
    \includegraphics[width=0.8\textwidth]
    {figures/altera_burst_impulse_18_18_error_polar_6.pdf}\\
    (f) Impulse (-0.41, -1.00)
    \end{center}
  \end{minipage}
  \caption{Altera Burst I/O FFT impulse error polar responses (first six).}
  \label{fig:altera_burst_impulse_error}
\end{figure}
%
% -----------------------------------------------------------------------------

% -----------------------------------------------------------------------------
% Radix-4 BFP FFT Polar Error Responses
% -----------------------------------------------------------------------------
%
\begin{figure}[p]
  \begin{minipage}{0.5\textwidth}
    \begin{center}
    \includegraphics[width=0.8\textwidth]
    {figures/radix4_bfp_impulse_18_18_error_polar_1.pdf}\\
    (a) Impulse (1.00, 0.00)
    \end{center}
  \end{minipage}
  \hfil
  \begin{minipage}{0.5\textwidth}
    \begin{center}
    \includegraphics[width=0.8\textwidth]
    {figures/radix4_bfp_impulse_18_18_error_polar_2.pdf}\\
    (b) Impulse (1.00, -0.41)
    \end{center}
  \end{minipage}
  \vskip5mm
  \begin{minipage}{0.5\textwidth}
    \begin{center}
    \includegraphics[width=0.8\textwidth]
    {figures/radix4_bfp_impulse_18_18_error_polar_3.pdf}\\
    (c) Impulse (1.00, -1.00)
    \end{center}
  \end{minipage}
  \hfil
  \begin{minipage}{0.5\textwidth}
    \begin{center}
    \includegraphics[width=0.8\textwidth]
    {figures/radix4_bfp_impulse_18_18_error_polar_4.pdf}\\
    (d) Impulse (0.41, -1.00)
    \end{center}
  \end{minipage}
  \vskip5mm
  \begin{minipage}{0.5\textwidth}
    \begin{center}
    \includegraphics[width=0.8\textwidth]
    {figures/radix4_bfp_impulse_18_18_error_polar_5.pdf}\\
    (e) Impulse (0.00, -1.00)
    \end{center}
  \end{minipage}
  \hfil
  \begin{minipage}{0.5\textwidth}
    \begin{center}
    \includegraphics[width=0.8\textwidth]
    {figures/radix4_bfp_impulse_18_18_error_polar_6.pdf}\\
    (f) Impulse (-0.41, -1.00)
    \end{center}
  \end{minipage}
  \caption{Radix-4 BFP FFT impulse error polar responses (first six).}
  \label{fig:radix4_bfp_impulse_error}
\end{figure}
%
% -----------------------------------------------------------------------------

\clearpage
% -----------------------------------------------------------------------------
% Exponent Codes Analysis
% -----------------------------------------------------------------------------
%
\begin{figure}[t]
  \begin{minipage}{0.5\textwidth}
    \begin{center}
    \includegraphics[width=0.95\textwidth]
    {figures/altera_burst_exp_codes_18_18_impulse.pdf}\\
    (a) Altera Burst I/O input impulse sweep
    \end{center}
  \end{minipage}
  \hfil
  \begin{minipage}{0.5\textwidth}
    \begin{center}
    \includegraphics[width=0.95\textwidth]
    {figures/radix4_bfp_exp_codes_18_18_impulse.pdf}\\
    (b) Radix-4 BFP input impulse sweep
    \end{center}
  \end{minipage}
  \vskip5mm
  \begin{minipage}{0.5\textwidth}
    \begin{center}
    \includegraphics[width=0.95\textwidth]
    {figures/altera_burst_exp_codes_18_18_constant.pdf}\\
    (c) Altera Burst I/O input constant sweep
    \end{center}
  \end{minipage}
  \hfil
  \begin{minipage}{0.5\textwidth}
    \begin{center}
    \includegraphics[width=0.95\textwidth]
    {figures/radix4_bfp_exp_codes_18_18_constant.pdf}\\
    (d) Radix-4 BFP input constant sweep
    \end{center}
  \end{minipage}
  \caption{Exponent codes measured using impulse and constant sweeps.}
  \label{fig:altera_burst_exp_codes}
\end{figure}
%
% -----------------------------------------------------------------------------

% -----------------------------------------------------------------------------
\subsection{Exponent Codes Analysis}
% -----------------------------------------------------------------------------

Figure~\ref{fig:altera_burst_exp_codes} shows the exponent codes generated
by the Altera Burst I/O and Radix-4 BFP FFTs for impulse and constant
inputs with real and imaginary amplitudes that are the maximum positive
value or negative value for the bit-width indicated on the $x$-axis.

Figure~\ref{fig:altera_burst_exp_codes} shows that the
Radix-4 BFP FFT operates as expected;
%
\begin{itemize}
\item \textbf{Impulse Inputs}

An impulse input does not have bit-growth through the FFT.
Figure~\ref{fig:altera_burst_exp_codes}(b) shows that for input impulses
that set any of the 3 MSBs of the input word, the exponent code is non-zero.

\item \textbf{Constant Inputs}

A constant input has a bit-growth of $\log_2(N)$ through the FFT.
For a 1024-point FFT, the bit-growth will be 10-bits. An input impulse
with a 17-bit magnitude will have a 27-bit output magnitude, so the
expected output exponent value is 10 or 11.
Figure~\ref{fig:altera_burst_exp_codes}(d) shows that for the 1024-point
FFT and the largest input, the exponent code is 11, and that as the
input amplitude is reduced, the exponent code also reduces.
\end{itemize}

\clearpage
Figure~\ref{fig:altera_burst_exp_codes} shows that the
Altera Burst I/O FFT does not operate as expected;
%
\begin{itemize}
\item \textbf{Impulse Inputs}

Figure~\ref{fig:altera_burst_exp_codes}(a) shows that for input impulses,
the exponent code is at least 2, and for large input signals, the
exponent code increases up to 6 for positive inputs and up to 5 for
negative inputs!

\item \textbf{Constant Inputs}

Figure~\ref{fig:altera_burst_exp_codes}(c) shows that for the 1024-point
FFT the full-scale input has an exponent code of 12 for positive inputs,
but an exponent code of 11 for negative inputs!

\end{itemize}
%
The Altera implementation of block floating-point appears to have two
implementation details that are not documented.
%
\begin{enumerate}
\item \textbf{The input is unconditionally scaled by 2-bits}

The Altera burst I/O FFT appears to scale the input by 2-bits regardless
of the input amplitude. This could be due to the Altera Burst I/O
FFT implementing the block exponent scaling after the radix-4 and twiddle
logic. Scaling after the radix-4 and twiddle logic requires the control
logic to track two scale factors; the scale factor being applied during
the current stage, and monitoring of the MSBs of the current stage
calculations to determine the scale factor to apply during the next stage.
Block floating-point scaling implemented after the radix-4 and twiddle
logic requires 5 guard bits, rather than 3 guard bits.

The Altera Burst I/O FFT should not require a 2-bit unconditional shift,
as the core could monitor the scaling over the 5 guard bits.

\item \textbf{Positive and negative inputs result in different scaling}

The cause of this discrepancy appears to be due to Altera's choice to
round up positive values, rather than round toward zero (or to saturate).

\begin{enumerate}
\item Full-scale positive input impulse
%
\begin{verbatim}
>> x = zeros(1024,1);x(1)=2^(Bx-1)-1;
>> X=altera_burst_fft_model(x,Bx,Bx,N,0)
       data: [1024x1 double]
   exponent: 6
\end{verbatim}
%
$(2^{(Bx-1)}-1) = \text{1FFFFh}$ with an exponent code of 6, implies a
binary point at\newline \verb+01_1111_1111_11.11_1111+ and the 12-bit
MSBs round up to the 13-bit number\newline \verb+0_1000_0000_0000 (800h)+,
where an additional bit was required to represent the sign. The 18-bit
number using during the next radix-4 pass is the sign-extended
value\newline\verb+00_0000_1000_0000_0000 (00800h)+ or 2048.

\item Negative of the full-scale positive input impulse
%
\begin{verbatim}
>> x = zeros(1024,1);x(1)=-(2^(Bx-1)-1);
>> X=altera_burst_fft_model(x,Bx,Bx,N,0)
      data: [1024x1 double]
  exponent: 5
\end{verbatim}
%
$-(2^{(Bx-1)}-1) = \text{20001h}$ with an exponent code of 5, implies a
binary point at\newline \verb+10_0000_0000_000.0_0001+ and the 12-bit
MSBs round to the 13-bit\newline \verb+1_0000_0000_0000 (1000h)+, where
the sign bit is still correct. The 18-bit number using during the next
radix-4 pass is the sign-extended value \verb+11_1111_0000_0000_0000 (3F00h)+ or -4096.
\end{enumerate}

The asymmetry in the exponent code response for positive and negative inputs
is caused by Altera's choice to round-up the positive values. If symmetric
rounding toward zero (or saturation) was used, then the exponent code
responses would be identical.
\end{enumerate}

\clearpage
% -----------------------------------------------------------------------------
\subsection{Complex-valued Exponential Input}
% -----------------------------------------------------------------------------

Figure~\ref{fig:altera_burst_exponential} shows plots of the FFT of
complex-valued exponentials for the Altera Burst I/O and Radix-4 BFP FFTs.
The MATLAB script generates 128 complex-valued sinusoids with uniformly
random center frequencies and phase, with full-scale magnitude.
%
Figures~\ref{fig:altera_burst_exponential}(a) and (b) show the magnitude
responses and hence the center frequencies.
%
Figures~\ref{fig:altera_burst_exponential}(c) and (d) show polar plots
of the center frequency magnitude (normalized to full-scale) and phase.
%
Figures~\ref{fig:altera_burst_exponential}(e) and (f) show polar plots
of the error between the fixed-point FFT and the MATLAB FFT.

The main difference between the Altera Burst I/O and Radix-4 FFT responses
for complex-valued exponential inputs was in the exponent codes;
the Altera Burst I/O generated exponent codes of 11 and 12, while the
Radix-4 BFP generated exponent codes of 9, 10, and 11.
%
This difference in exponent code values helps interpret the error responses in
Figures~\ref{fig:altera_burst_exponential}(e) and (f). The Radix-4 BFP FFT
errors appear larger than those of the Altera Burst I/O FFT because the
exponent codes of the Radix-2 response are smaller by 2. The Radix-4 BFP
error response needs to be divided-by-4 (7.90/4 = 1.98) to be on the same
scale as the Altera Burst I/O FFT error response (1.89).

\textcolor{magenta}{The Altera Burst I/O FFT of full-scale complex-valued
exponential inputs generates exponent codes 1 or 2 higher than the Radix-4 BFP FFT}.

% -----------------------------------------------------------------------------
% Complex-valued Exponential Responses
% -----------------------------------------------------------------------------
%
\begin{figure}[p]
  \begin{minipage}{0.5\textwidth}
    \begin{center}
    \includegraphics[width=\textwidth]
    {figures/altera_burst_exponential_18_18_channels.pdf}\\
    (a) Altera Burst I/O frequency channels
    \end{center}
  \end{minipage}
  \hfil
  \begin{minipage}{0.5\textwidth}
    \begin{center}
    \includegraphics[width=\textwidth]
    {figures/radix4_bfp_exponential_18_18_channels.pdf}\\
    (b) Radix-4 BFP frequency channels
    \end{center}
  \end{minipage}
  \vskip5mm
  \begin{minipage}{0.5\textwidth}
    \begin{center}
    \includegraphics[width=\textwidth]
    {figures/altera_burst_exponential_18_18_polar.pdf}\\
    (c) Altera Burst I/O frequency channel phase
    \end{center}
  \end{minipage}
  \hfil
  \begin{minipage}{0.5\textwidth}
    \begin{center}
    \includegraphics[width=\textwidth]
    {figures/radix4_bfp_exponential_18_18_polar.pdf}\\
    (d) Radix-4 BFP frequency channel phase
    \end{center}
  \end{minipage}
  \vskip5mm
  \begin{minipage}{0.5\textwidth}
    \begin{center}
    \includegraphics[width=\textwidth]
    {figures/altera_burst_exponential_18_18_error.pdf}\\
    (e) Altera Burst I/O error
    \end{center}
  \end{minipage}
  \hfil
  \begin{minipage}{0.5\textwidth}
    \begin{center}
    \includegraphics[width=\textwidth]
    {figures/radix4_bfp_exponential_18_18_error.pdf}\\
    (f) Radix-4 BFP error
    \end{center}
  \end{minipage}
  \caption{FFT spectral responses for complex-valued exponential input.}
  \label{fig:altera_burst_exponential}
\end{figure}
%
% -----------------------------------------------------------------------------

\clearpage
% -----------------------------------------------------------------------------
% Uniform Noise Responses
% -----------------------------------------------------------------------------
%
\begin{figure}[t]
  \begin{minipage}{0.5\textwidth}
    \begin{center}
    \includegraphics[width=\textwidth]
    {figures/altera_burst_uniform_18_18_spectra.pdf}\\
    (a) Altera Burst I/O
    \end{center}
  \end{minipage}
  \hfil
  \begin{minipage}{0.5\textwidth}
    \begin{center}
    \includegraphics[width=\textwidth]
    {figures/radix4_bfp_uniform_18_18_spectra.pdf}\\
    (b) Radix-4 BFP
    \end{center}
  \end{minipage}
  \caption{FFT spectral responses for uniform noise input.}
  \label{fig:altera_burst_uniform}
\end{figure}
%
% -----------------------------------------------------------------------------

% -----------------------------------------------------------------------------
\subsection{Uniform Noise Input}
% -----------------------------------------------------------------------------

Figure~\ref{fig:altera_burst_uniform} shows that the Altera Burst I/O and
Radix-4 BFP FFTs have similar spectral performance for (full-scale) uniform
noise input. The main difference between the two FFTs was in the exponent
codes; the Altera Burst I/O generated exponent codes of 8 and 9, while the
Radix-4 BFP generated exponent codes of 7 and 8.

\textcolor{magenta}{The Altera Burst I/O FFT of full-scale complex-valued
uniform noise inputs generates exponent codes 1 higher than the Radix-4 BFP FFT}.

\clearpage
% -----------------------------------------------------------------------------
% Gaussian Noise Responses
% -----------------------------------------------------------------------------
%
\begin{table}
\caption{FFT Noise Power Ratio (NPR) for Gaussian noise input.}
\label{tab:altera_burst_gaussian_npr}
\begin{center}
\begin{tabular}{|c||c|c|c||c|c|c|}
\hline
        & \multicolumn{3}{c|}{Altera Burst I/O}
        & \multicolumn{3}{c|}{Radix-4 BFP}\\
\cline{2-7}
LF (dB) & QN (dB) & NPR (dB) & Exp. Codes & QN (dB) & NPR (dB) & Exp. Codes\\
\hline\hline
&&&&&&\\
-15 & -88.5 & 73.5 & 7,8 &  -88.1 & 73.1 & 6,7\\
-20 & -90.6 & 70.6 & 6,7 &  -92.1 & 72.1 & 5,6\\
-25 & -92.1 & 67.1 & 5,6 &  -95.2 & 70.2 & 4,5\\
-30 & -93.6 & 63.6 & 5   & -100.8 & 70.8 & 3,4\\
-35 & -93.9 & 58.9 & 4,5 & -105.7 & 70.7 & 3\\
-40 & -94.0 & 54.0 & 3,4 & -107.9 & 67.9 & 2,3\\
-45 & -94.1 & 49.1 & 2,3 & -108.4 & 63.4 & 1,2\\
-50 & -94.1 & 44.1 & 2   & -108.5 & 58.5 & 0,1\\
&&&&&&\\
\hline
\end{tabular}
\end{center}
\end{table}

% -----------------------------------------------------------------------------
\subsection{Gaussian Noise Input}
% -----------------------------------------------------------------------------

Table~\ref{tab:altera_burst_gaussian_npr} and
Figure~\ref{fig:altera_burst_gaussian} show the FFT spectral responses for
Gaussian noise input.
%
The Loading Factor (LF) is the RMS power of the input signal relative to full-scale,
eg., a loading factor of -12dB corresponds to an RMS of quarter full-scale.
%
The Noise Power Ratio (NPR) is the difference between the input Loading Factor
and the Quantization Noise (QN).
%
Table~\ref{tab:altera_burst_gaussian_npr} shows the QN, and NPR for a range
of input LFs.
Figure~\ref{fig:altera_burst_gaussian} show the FFT spectral responses for
LFs of -15dB, -25dB, and -35dB.

\textcolor{magenta}{The Altera Burst I/O FFT has a quantization noise floor
that is 14dB (about 2-bits) higher than the Radix-4 BFP FFT}.

% -----------------------------------------------------------------------------
\subsection{Bandpass Noise Input}
% -----------------------------------------------------------------------------

Figure~\ref{fig:altera_burst_bandpass} show the FFT spectral responses for
bandpass noise input. The bandpass signal was generated from complex-valued
baseband Gaussian noise.
%
Figure~\ref{fig:altera_burst_bandpass} show the FFT spectral responses for
LFs of -20dB, -30dB, and -40dB.

\textcolor{magenta}{The Altera Burst I/O FFT has a quantization noise floor
that is about 2-bits higher than the Radix-4 BFP FFT}.

% -----------------------------------------------------------------------------
\subsection{Notch-filtered Noise Input}
% -----------------------------------------------------------------------------

Figure~\ref{fig:altera_burst_notch} show the FFT spectral responses for
notch-filtered noise input. The notch-filtered signal was generated from
complex-valued baseband Gaussian noise.
%
Figure~\ref{fig:altera_burst_notch} show the FFT spectral responses for
LFs of -20dB, -30dB, and -40dB.

\textcolor{magenta}{The Altera Burst I/O FFT has a quantization noise floor
that is about 2-bits higher than the Radix-4 BFP FFT}.

% -----------------------------------------------------------------------------
% Gaussian Noise Responses
% -----------------------------------------------------------------------------
%
\begin{figure}[p]
  \begin{minipage}{0.5\textwidth}
    \begin{center}
    \includegraphics[width=\textwidth]
    {figures/altera_burst_gaussian_18_18_n15dB_spectra.pdf}\\
    (a) Altera Burst I/O, LF = -15dB
    \end{center}
  \end{minipage}
  \hfil
  \begin{minipage}{0.5\textwidth}
    \begin{center}
    \includegraphics[width=\textwidth]
    {figures/radix4_bfp_gaussian_18_18_n15dB_spectra.pdf}\\
    (b) Radix-4 BFP, LF = -15dB
    \end{center}
  \end{minipage}
  \vskip5mm
  \begin{minipage}{0.5\textwidth}
    \begin{center}
    \includegraphics[width=\textwidth]
    {figures/altera_burst_gaussian_18_18_n25dB_spectra.pdf}\\
    (c) Altera Burst I/O, LF = -25dB
    \end{center}
  \end{minipage}
  \hfil
  \begin{minipage}{0.5\textwidth}
    \begin{center}
    \includegraphics[width=\textwidth]
    {figures/radix4_bfp_gaussian_18_18_n25dB_spectra.pdf}\\
    (d) Radix-4 BFP, LF = -25dB
    \end{center}
  \end{minipage}
  \vskip5mm
  \begin{minipage}{0.5\textwidth}
    \begin{center}
    \includegraphics[width=\textwidth]
    {figures/altera_burst_gaussian_18_18_n35dB_spectra.pdf}\\
    (e) Altera Burst I/O, LF = -35dB
    \end{center}
  \end{minipage}
  \hfil
  \begin{minipage}{0.5\textwidth}
    \begin{center}
    \includegraphics[width=\textwidth]
    {figures/radix4_bfp_gaussian_18_18_n35dB_spectra.pdf}\\
    (f) Radix-4 BFP, LF = -35dB
    \end{center}
  \end{minipage}
  \caption{FFT spectral responses for Gaussian noise input.}
  \label{fig:altera_burst_gaussian}
\end{figure}
%
% -----------------------------------------------------------------------------

% -----------------------------------------------------------------------------
% Bandpass Noise Responses
% -----------------------------------------------------------------------------
%
\begin{figure}[p]
  \begin{minipage}{0.5\textwidth}
    \begin{center}
    \includegraphics[width=\textwidth]
    {figures/altera_burst_bandpass_18_18_n20dB_spectra.pdf}\\
    (a) Altera Burst I/O, LF = -20dB
    \end{center}
  \end{minipage}
  \hfil
  \begin{minipage}{0.5\textwidth}
    \begin{center}
    \includegraphics[width=\textwidth]
    {figures/radix4_bfp_bandpass_18_18_n20dB_spectra.pdf}\\
    (b) Radix-4 BFP, LF = -20dB
    \end{center}
  \end{minipage}
  \vskip5mm
  \begin{minipage}{0.5\textwidth}
    \begin{center}
    \includegraphics[width=\textwidth]
    {figures/altera_burst_bandpass_18_18_n30dB_spectra.pdf}\\
    (c) Altera Burst I/O, LF = -30dB
    \end{center}
  \end{minipage}
  \hfil
  \begin{minipage}{0.5\textwidth}
    \begin{center}
    \includegraphics[width=\textwidth]
    {figures/radix4_bfp_bandpass_18_18_n30dB_spectra.pdf}\\
    (d) Radix-4 BFP, LF = -30dB
    \end{center}
  \end{minipage}
  \vskip5mm
  \begin{minipage}{0.5\textwidth}
    \begin{center}
    \includegraphics[width=\textwidth]
    {figures/altera_burst_bandpass_18_18_n40dB_spectra.pdf}\\
    (e) Altera Burst I/O, LF = -40dB
    \end{center}
  \end{minipage}
  \hfil
  \begin{minipage}{0.5\textwidth}
    \begin{center}
    \includegraphics[width=\textwidth]
    {figures/radix4_bfp_bandpass_18_18_n40dB_spectra.pdf}\\
    (f) Radix-4 BFP, LF = -40dB
    \end{center}
  \end{minipage}
  \caption{FFT spectral responses for bandpass noise input.}
  \label{fig:altera_burst_bandpass}
\end{figure}
%
% -----------------------------------------------------------------------------

% -----------------------------------------------------------------------------
% Notch-Filterd Noise Responses
% -----------------------------------------------------------------------------
%
\begin{figure}[p]
  \begin{minipage}{0.5\textwidth}
    \begin{center}
    \includegraphics[width=\textwidth]
    {figures/altera_burst_notch_18_18_n20dB_spectra.pdf}\\
    (a) Altera Burst I/O, LF = -20dB
    \end{center}
  \end{minipage}
  \hfil
  \begin{minipage}{0.5\textwidth}
    \begin{center}
    \includegraphics[width=\textwidth]
    {figures/radix4_bfp_notch_18_18_n20dB_spectra.pdf}\\
    (b) Radix-4 BFP, LF = -20dB
    \end{center}
  \end{minipage}
  \vskip5mm
  \begin{minipage}{0.5\textwidth}
    \begin{center}
    \includegraphics[width=\textwidth]
    {figures/altera_burst_notch_18_18_n30dB_spectra.pdf}\\
    (c) Altera Burst I/O, LF = -30dB
    \end{center}
  \end{minipage}
  \hfil
  \begin{minipage}{0.5\textwidth}
    \begin{center}
    \includegraphics[width=\textwidth]
    {figures/radix4_bfp_notch_18_18_n30dB_spectra.pdf}\\
    (d) Radix-4 BFP, LF = -30dB
    \end{center}
  \end{minipage}
  \vskip5mm
  \begin{minipage}{0.5\textwidth}
    \begin{center}
    \includegraphics[width=\textwidth]
    {figures/altera_burst_notch_18_18_n40dB_spectra.pdf}\\
    (e) Altera Burst I/O, LF = -40dB
    \end{center}
  \end{minipage}
  \hfil
  \begin{minipage}{0.5\textwidth}
    \begin{center}
    \includegraphics[width=\textwidth]
    {figures/radix4_bfp_notch_18_18_n40dB_spectra.pdf}\\
    (f) Radix-4 BFP, LF = -40dB
    \end{center}
  \end{minipage}
  \caption{FFT spectral responses for notch noise input.}
  \label{fig:altera_burst_notch}
\end{figure}
%
% -----------------------------------------------------------------------------

\clearpage
% -----------------------------------------------------------------------------
% NPR
% -----------------------------------------------------------------------------
%
\begin{figure}[t]
  \begin{minipage}{0.5\textwidth}
    \begin{center}
    \includegraphics[width=\textwidth]
    {figures/altera_burst_npr_18_18.pdf}\\
    (a) Altera Burst I/O NPR
    \end{center}
  \end{minipage}
  \hfil
  \begin{minipage}{0.5\textwidth}
    \begin{center}
    \includegraphics[width=\textwidth]
    {figures/radix4_bfp_npr_18_18.pdf}\\
    (b) Radix-4 BFP NPR
    \end{center}
  \end{minipage}
  \vskip5mm
  \begin{minipage}{0.5\textwidth}
    \begin{center}
    \includegraphics[width=\textwidth]
    {figures/altera_burst_npr_18_18_exp_codes.pdf}\\
    (c) Altera Burst I/O NPR exponent codes
    \end{center}
  \end{minipage}
  \hfil
  \begin{minipage}{0.5\textwidth}
    \begin{center}
    \includegraphics[width=\textwidth]
    {figures/radix4_bfp_npr_18_18_exp_codes.pdf}\\
    (d) Radix-4 BFP NPR exponent codes
    \end{center}
  \end{minipage}
  \caption{Noise Power Ratio measured using notch-filtered noise.}
  \label{fig:altera_burst_npr}
\end{figure}
%
% -----------------------------------------------------------------------------

% -----------------------------------------------------------------------------
\subsection{Noise Power Ratio (NPR)}
% -----------------------------------------------------------------------------

Figure~\ref{fig:altera_burst_npr} shows the Noise Power Ratio (NPR) measured
for the Altera Burst I/O FFT and the Radix-4 BFP FFT. The NPR was measured
using notch-filtered noise (see Figure~\ref{fig:altera_burst_notch}) for
a range of input loading factors.

The NPR for the Radix-4 BFP in Figures~\ref{fig:altera_burst_npr}(b) and (d)
show how the BFP implements automatic gain control (AGC), preserving the
signal-to-noise at a constant level (73dB) for a wide range of input powers
(loading factors of -40dB to -20dB).
%
The NPR for the Altera Burst I/O FFT in Figures~\ref{fig:altera_burst_npr}(a)
and (c) \emph{does not} show the same AGC response as the Radix-4 BFP.
The Altera Burst FFT consistently has 2-bits less signal to noise relative
to the MATLAB FFT (of the fixed-point input data) and the exponent codes
are 2 higher than the Radix-4 BFP exponent codes.

\textcolor{magenta}{The Altera Burst I/O FFT NPR is about 2-bits lower than the
Radix-4 BFP FFT (for input loading factors of less than -20dB)}.
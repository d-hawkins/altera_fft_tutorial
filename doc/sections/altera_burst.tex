% =============================================================================
\section{Altera Burst I/O FFT}
% =============================================================================
\label{sec:burst_fft}

% -----------------------------------------------------------------------------
\subsection{Quartus IP Generation}
% -----------------------------------------------------------------------------

\begin{enumerate}
\item Start Quartus.

The version used to develop this tutorial was:
\begin{verbatim}
tcl> puts $quartus(version)
Version 22.1std.1 Build 917 02/14/2023 SC Lite Edition
\end{verbatim}
%$
\item Open the FFT IP core parameter editor.
%
\begin{itemize}
\item Select the IP Catalog search tool.
\item Find the FFT IP core.
\item Double-mouse-click the FFT core to open the parameter editor.
\end{itemize}
%
\item Configure the FFT IP variation.
%
\begin{itemize}
\item \textbf{Create IP Variation}
\begin{itemize}
\item Entity name: \texttt{altera\_burst\_fft}
\item Save in the directory:\\
\texttt{C:/github/altera\_fft\_tutorial/designs/burst/build/altera\_burst\_fft}
\end{itemize}
%
\item \textbf{Target Device}
\begin{itemize}
\item Family: Cyclone V
\item Device: 5CEBA2F17A7
\end{itemize}
\item Click the \emph{OK} button.
\end{itemize}
%
\item Configure the FFT IP parameters.
%
\begin{itemize}
\item \textbf{Transform}
\begin{itemize}
\item Length: 1024
\item Direction: Bi-directional
\end{itemize}
%
\item \textbf{I/O}
\begin{itemize}
\item Data Flow: Burst
\item Input Order: Natural
\item Output Order: Natural
\end{itemize}
%
\item \textbf{Data and Twiddle}
\begin{itemize}
\item Representation: Block Floating Point
\item Data Input Width: 16 bits
\item Twiddle Width: 16 bits
\item Data Output Width: 16 bits
\end{itemize}
%
\item Click the menu option \emph{File}$\rightarrow$\emph{Save}, and then
click the save dialog \emph{Close} button.
\end{itemize}
%
The build directory then contains the Qsys file
\texttt{build/altera\_burst\_fft.qsys},
along with the directory \texttt{build/altera\_burst\_fft} that contains
files, but no subdirectories.
%
% -----------------------------------------------------------------------------
% Altera Burst FFT
% -----------------------------------------------------------------------------
%
\begin{figure}[p]
  \begin{center}
    \includegraphics[width=\textwidth]
    {figures/altera_burst_fft_ip_parameter_editor.png}
  \end{center}
  \caption{Altera Burst FFT IP Parameter Editor view.}
  \label{fig:altera_burst_fft_ip_parameter_editor}
\end{figure}
% -----------------------------------------------------------------------------
%
Figure~\ref{fig:altera_burst_fft_ip_parameter_editor} shows the IP Parameter
Editor view.

%
\newpage
\item Generate the synthesis and simulation code.
%
\begin{itemize}
\item Click the menu option \emph{Generate}$\rightarrow$\emph{Generate HDL}.
\item Under the \textbf{Simulation} section change the simulation model
pull-down to Verilog.
\item The other settings do not need to be changed.
\item Click the \emph{Generate} button, and then click the
generate dialog \emph{Close} button.
\end{itemize}
%
The build directory now contains:
\begin{itemize}
\item \texttt{build/altera\_burst\_fft/synthesis}
\item \texttt{build/altera\_burst\_fft/simulation}
\end{itemize}

\item Generate the Testbench System simulation code.
%
\begin{itemize}
\item Click the menu option \emph{Generate}$\rightarrow$\emph{Generate
Testbench System}.
\item The default settings do not need to be changed.
\item Click the \emph{Generate} button and then click the
generate dialog \emph{Close} button.
\end{itemize}
%
The build directory now contains:
\begin{itemize}
\item \texttt{build/altera\_burst\_fft/testbench}
\end{itemize}

\item Generate the example design.
%
\begin{itemize}
\item Click the menu option \emph{Generate}$\rightarrow$\emph{Generate
Example Design}$\rightarrow$\emph{fft\_ii\_0}.
\item Change the output directory to:\\
\texttt{C:/github/altera\_fft\_tutorial/designs/burst/build/altera\_burst\_fft\_example}
\item Click the \emph{OK} button, and then click the
generate dialog \emph{Close} button.
\end{itemize}
%
The build directory now contains:
\begin{itemize}
\item \texttt{build/altera\_burst\_fft\_example}
\end{itemize}

%
\item Click the \emph{Finish} button.
\item Exit Quartus.
\end{enumerate}

\clearpage
% -----------------------------------------------------------------------------
\subsection{Questasim Simulation}
% -----------------------------------------------------------------------------

\begin{enumerate}
\item Start Questasim.

The version used to develop this tutorial was the Questa Intel Starter
Edition version:
%
\begin{verbatim}
Questa> vsim -version
# Questa  Intel Starter FPGA Edition-64 vsim 2021.2 Simulator 2021.04
  Apr 14 2021
\end{verbatim}
%
\item Change to the mentor simulation script directory.
%
\begin{verbatim}
Questa> cd C:/github/altera_fft_tutorial/designs/burst/build/
  altera_burst_fft_example/simulation_scripts/mentor
\end{verbatim}
%$
\item Source the simulation script.
%
\begin{verbatim}
Questa> source msim_setup.tcl
\end{verbatim}
%
\item Build the example design.
%
\begin{verbatim}
Questa> ld_debug
\end{verbatim}
%
\item Add the FFT core signals to the waveform window.
%
\begin{itemize}
\item Expand the design hierarchy and select\\
\texttt{test\_program/tb/fft\_ii\_0\_example\_design\_inst/core}
\item Right-mouse-click and select \emph{Add Wave}
\end{itemize}
%
The Tcl console will contain the equivalent Tcl command
%
\begin{verbatim}
add wave -position insertpoint sim:/test_program/tb/
  fft_ii_0_example_design_inst/core/*
\end{verbatim}
%
\item Run the simulation.
%
\begin{verbatim}
Questa> run -a
\end{verbatim}
%
When the \emph{Finish Vsim} dialog appears, click \emph{No}, so that
Questasim does not exit.
%
\item View the simulation waveforms.

Figure~\ref{fig:altera_burst_fft_example_waveforms} shows the example design
waveforms. Waveform dividers were added to show the FFT input (sink) and FFT
output (source) signals. The FFT core input \texttt{ready} was highlighted in
cyan to show when the core was ready for input data, while the core
output \texttt{valid} was highlighted in pink to show then the output
data was valid.

The example design testbench does not implement any \emph{verification}, i.e.,
it is not a self-checking testbench that compares the FFT core output with
an expected output. The example design testbench generates FFT core input data
by reading data from real and imaginary files, and it writes output data
to real, imaginary, and exponent files.

The example design can be used to understand the FFT core, and as the basis
to develop a self-checking testbench.
%
\item Exit Questasim.
\end{enumerate}

% -----------------------------------------------------------------------------
% Altera Burst FFT
% -----------------------------------------------------------------------------
%
\begin{landscape}
\begin{figure}[p]
  \begin{center}
    \includegraphics[width=210mm]
    {figures/altera_burst_fft_example_waveforms.png}
  \end{center}
  \caption{Altera Burst FFT example design waveforms.}
  \label{fig:altera_burst_fft_example_waveforms}
\end{figure}
\end{landscape}
% -----------------------------------------------------------------------------

\clearpage
% -----------------------------------------------------------------------------
\subsection{MATLAB Model}
% -----------------------------------------------------------------------------


\begin{enumerate}
\item Start MATLAB.

The version used to develop this tutorial was R2023b:
%
\begin{verbatim}
>> version
   '23.2.0.2485118 (R2023b) Update 6'
\end{verbatim}
%
\item Change to the MATLAB simulation script directory.
%
\begin{verbatim}
>> cd 'C:/github/altera_fft_tutorial/designs/burst/build/
   altera_burst_fft_example/Matlab_model'
\end{verbatim}

\item Copy the mentor input stimulus files to the MATLAB directory.
%
\begin{verbatim}
fft_ii_0_example_design_real_input.txt
fft_ii_0_example_design_imag_input.txt
\end{verbatim}

\item Run the example design testbench script.
%
\begin{verbatim}
>> fft_ii_0_example_design_tb
\end{verbatim}
%
The script generates three output files
%
\begin{verbatim}
fft_ii_0_example_design_real_output_c_model.txt
fft_ii_0_example_design_imag_output_c_model.txt
fft_ii_0_example_design_exponent_out_c_model.txt
\end{verbatim}

\item Compare the MATLAB output files to those in the mentor directory.

The mentor directory filenames are:
%
\begin{verbatim}
fft_ii_0_example_design_real_output.txt
fft_ii_0_example_design_imag_output.txt
fft_ii_0_example_design_exponent_out.txt
\end{verbatim}
%
The files all match (as long as minor whitespace differences are ignored).
%
\item Exit MATLAB.
\end{enumerate}
%
The MATLAB testbench demonstrates the use of the MATLAB model.